%%%%%%%%%%%%%%%%%%%%%%%%%%%%%%%%%%%%%%%%%
% Thin Sectioned Essay
% LaTeX Template
% Version 1.0 (3/8/13)
%
% This template has been downloaded from:
% http://www.LaTeXTemplates.com
%
% Original Author:
% Nicolas Diaz (nsdiaz@uc.cl) with extensive modifications by:
% Vel (vel@latextemplates.com)
%
% License:
% CC BY-NC-SA 3.0 (http://creativecommons.org/licenses/by-nc-sa/3.0/)
%
%%%%%%%%%%%%%%%%%%%%%%%%%%%%%%%%%%%%%%%%%

%----------------------------------------------------------------------------------------
%	PACKAGES AND OTHER DOCUMENT CONFIGURATIONS
%----------------------------------------------------------------------------------------

\documentclass[a4paper, 11pt]{article} % Font size (can be 10pt, 11pt or 12pt) and paper size (remove a4paper for US letter paper)

\usepackage[protrusion=true,expansion=true]{microtype} % Better typography
\usepackage{graphicx} % Required for including pictures
\usepackage{wrapfig} % Allows in-line images

\usepackage{mathpazo} % Use the Palatino font
\usepackage[T1]{fontenc} % Required for accented characters
\linespread{1.05} % Change line spacing here, Palatino benefits from a slight increase by default

%%%%%%% Custom
\usepackage{hyperref}
\hypersetup{
    colorlinks=true,
    linkcolor=blue,
    filecolor=magenta,      
    urlcolor=cyan,
}

\makeatletter
\renewcommand\@biblabel[1]{\textbf{#1.}} % Change the square brackets for each bibliography item from '[1]' to '1.'
\renewcommand{\@listI}{\itemsep=0pt} % Reduce the space between items in the itemize and enumerate environments and the bibliography

\renewcommand{\maketitle}{ % Customize the title - do not edit title and author name here, see the TITLE block below
\begin{flushright} % Right align
{\LARGE\@title} % Increase the font size of the title

\vspace{50pt} % Some vertical space between the title and author name

{\large\@author} % Author name
\\\@date % Date

\vspace{40pt} % Some vertical space between the author block and abstract
\end{flushright}
}

%----------------------------------------------------------------------------------------
%	TITLE
%----------------------------------------------------------------------------------------

\title{\textbf{Protocol Buffers}\\ % Title
Chat Application Design} % Subtitle

\author{\textsc{Nicholas Mahlangu} % Author
\\\textsc{Tianyu Liu}
\\{\textit{Harvard College}}} % Institution

\date{\today} % Date

%----------------------------------------------------------------------------------------

\begin{document}

\maketitle % Print the title section

%----------------------------------------------------------------------------------------
%	ABSTRACT AND KEYWORDS
%----------------------------------------------------------------------------------------

%\renewcommand{\abstractname}{Summary} % Uncomment to change the name of the abstract to something else

%\begin{abstract}
%Morbi tempor congue porta. Proin semper, leo vitae faucibus dictum, metus mauris lacinia lorem, ac congue leo felis eu turpis. Sed nec nunc pellentesque, gravida eros at, porttitor ipsum. Praesent consequat urna a lacus lobortis ultrices eget ac metus. In tempus hendrerit rhoncus. Mauris dignissim turpis id sollicitudin lacinia. Praesent libero tellus, fringilla nec ullamcorper at, ultrices id nulla. Phasellus placerat a tellus a malesuada.
%\end{abstract}
%
%\hspace*{3,6mm}\textit{Keywords:} lorem , ipsum , dolor , sit amet , lectus % Keywords
%
%\vspace{30pt} % Some vertical space between the abstract and first section

%----------------------------------------------------------------------------------------
%	ESSAY BODY
%----------------------------------------------------------------------------------------

\section*{Introduction}

\section*{Interfaces Being Looked At}

%Interfaces: For our RESTful application, a server and client communicate over HTTP get and post requests. On the client side, these requests are decoded in JavaScript; on the server side, these requests are decoded in Python; they are then dispatched as select, update, or delete SQL requests.

The client and server communicate over HTTP by sending serialized data. All requests on both the client and server are encoded / decoded using JavaScript, as Meteor runs the same JavaScript code for both parties. 

\section*{Assumed Environment}

%Environment: For our RESTful application, the client's environment must have access to HTTP passed over port 8080 and must be able to run JavaScript. The server must likewise be able to access HTTP passed over port 8080, via an HTTPServer library. The server must also run Python.

The client and server must have access to HTTP passed over a port (3000 is the default one we use) and must be able to run JavaScript. Furthermore, the server needs to have \href{https://www.npmjs.com/}{npm} and \href{https://www.meteor.com/}{Meteor} installed.

\section*{System Design}

%\section*{Tianyu's Transcription}
%We want to replace JSON communication layer with protocol buffers. IN order to do that, we dig down into Meteor's implementation. There's  a package called ddp-common that handle's communication across sockets. we want to replace JSON communication layer with protocol buffers by replacing 2 methods inside utility.js. Those 2 methods are parse\_ddp() and stringify\_ddp(), which currently only rely on JSON. There is a very major complication here because the entire Meteor framework will assume that objects being passed around through sockets are JSON objects, so we actually need to have another layer of abstraction that takes a JSON object and converts that into a protocol buffer object, and then we serialize it. The implementation for that part is being done in json\_protocol.js. In json\_protocol.js, we manually define the protocol buffer abstraction for Javascript objects, which includes the majority of basic types, including integer, double, string, boolean, null, array, and object; array and object can actually have nested properties within. One thing that's annoying is that we cannot use npm\_require for Meteor because npm\_require will fail on the front end. Since ddp-common is a package that will be required on the front-end and the back-end, we will need to find a way to include all the packages without relying on npm\_require. That's non-trivial because ProtocolBufferJS is actually wrapped as a Node module. The way we solved this is take down the source files, manually import them into the directory, and remove all the npm\_require statements. So we pretty much just have a manual dependency tree by including files in a certain sequence. 
%
%If we were to do this again, we would actually use Node.js because so many of the functions rely on the objects being JSON objects, so our freedom is actually very limited. 
%
%All the implementation is inside utility.js and you can see that when it's deserializing, it's literally taking the string message and calling json\_protocol\_parse(); when serializing, you do json\_protoify(copy), where copy is a copy of the original question. After you serialize the entire thing, the entire thing will be converted into a hex string, meaning the actual data being sent over the wire is much, much smaller. This is especially useful when doing intense database operations like creating new lists. 
%
%TODO: search by user wildcard (Tianyu will do it after midnight), list users on the page where you create a new group, alert() that pops up when you mess up creating a new group.
%
%It's worth serializing on the bottom level because the operation you do at every single step will be converted into a comprehensive message package that contains like who's doing the action and what kind of action is it. 

Our system was mainly built using \textit{Meteor}, the popular JavaScript app framework, because the same code runs on the client and the server, it transfers really easily to iOS and Android, and has a highly reactive GUI thanks to the Blaze framework. In order to integrate Protocol Buffers into \textit{Meteor}, we had to dig into it's implementation files and modify the package, called \textit{ddp-common}, responsible for \textit{Meteor's} JSON communication. In this package is a file called \textit{utils.js} which contains the functions \textbf{parseDDP()} and \textbf{stringifyDDP()}, both of which only rely on JSON. The implication here is that the entire Meteor framework assumes objects are being passed around as JSON objects, so we had to create another layer of abstraction that takes a JSON object, converts it into a Protocol Buffer object, and then serializes it. The implementation for this is done in \textbf{json\_proto.js}: we manually define the Protocol Buffer abstraction for JavaScript objects, including the basics types (integer, double, string, boolean, etc.), arrays, and objects (the latter 2 can have nested properties within). 

One snag we ran into is that we couldn't use npm's \textbf{require()} statements, because Meteor would ignore them on the front-end. Since \textit{ddp-common} is used on the front-end and the back-end, we had to find a way to include all relevant packages without relying on \textbf{require()}. The way we did this is by saving all the source files, manually importing them into the directory, and removing all \textbf{require()} statements in our code. In essence, we created a manual dependency tree by including files in a certain sequence. 

Back to \textit{utils.js}, we see serialization now happening with our \textbf{JSONproto.protoify()} and deserialization with \textbf{JSONproto.parse()}. After serialization, the data is converted to a hex string, meaning the data being sent over the wire is much, much smaller. This is incredibly useful when doing intense database operations and one of the main benefits of using Protocol Buffers.

\section*{Resources Conserved}

\textit{Meteor} relies on 2 technologies for keeping the DOM up-to-date with the server: \textit{Blaze}, a declarative library for creating live-updating, reactive user experiences; and \textit{DDP}, a simple protocol for fetching structured data from a server and receiving live updates when data changes (basically REST for web sockets). Because these technologies are built straight into \textit{Meteor}, we minimize the number of requests flowing between the client and the server.

Furthermore, Protocol Buffers offer us a number of other benefits. When compared to XML, Protocol Buffers are 3 - 10 times smaller and 20 to 100 times faster, so we minimize the size of the data we're sending between the client and server and are able to do it much faster.

\section*{Resources Wasted}

\section*{Failure Conditions}




%This statement requires citation \cite{Smith:2012qr}; this one does too \cite{Smith:2013jd}. Lorem ipsum dolor sit amet, consectetur adipiscing elit. Aenean dictum lacus sem, ut varius ante dignissim ac. Sed a mi quis lectus feugiat aliquam. Nunc sed vulputate velit. Sed commodo metus vel felis semper, quis rutrum odio vulputate. Donec a elit porttitor, facilisis nisl sit amet, dignissim arcu. Vivamus accumsan pellentesque nulla at euismod. Duis porta rutrum sem, eu facilisis mi varius sed. Suspendisse potenti. Mauris rhoncus neque nisi, ut laoreet augue pretium luctus. Vestibulum sit amet luctus sem, luctus ultrices leo. Aenean vitae sem leo.
%
%Nullam semper quam at ante convallis posuere. Ut faucibus tellus ac massa luctus consectetur. Nulla pellentesque tortor et aliquam vehicula. Maecenas imperdiet euismod enim ut pharetra. Suspendisse pulvinar sapien vitae placerat pellentesque. Nulla facilisi. Aenean vitae nunc venenatis, vehicula neque in, congue ligula.
%
%Pellentesque quis neque fringilla, varius ligula quis, malesuada dolor. Aenean malesuada urna porta, condimentum nisl sed, scelerisque nisi. Suspendisse ac orci quis massa porta dignissim. Morbi sollicitudin, felis eget tristique laoreet, ante lacus pretium lacus, nec ornare sem lorem a velit. Pellentesque eu erat congue, ullamcorper ante ut, tristique turpis. Nam sodales mi sed nisl tincidunt vestibulum. Interdum et malesuada fames ac ante ipsum primis in faucibus.



%------------------------------------------------

%\section*{Section Name}

%Cras gravida, est vel interdum euismod, tortor mi lobortis mi, quis adipiscing elit lacus ut orci. Phasellus nec fringilla nisi, ut vestibulum neque. Aenean non risus eu nunc accumsan condimentum at sed ipsum.
%\begin{wrapfigure}{l}{0.4\textwidth} % Inline image example
%\begin{center}
%\includegraphics[width=0.38\textwidth]{fish.png}
%\end{center}
%\caption{Fish}
%\end{wrapfigure}
%Aliquam fringilla non diam sed varius. Suspendisse tellus felis, hendrerit non bibendum ut, adipiscing vitae diam. Lorem ipsum dolor sit amet, consectetur adipiscing elit. Nulla lobortis purus eget nisl scelerisque, commodo rhoncus lacus porta. Vestibulum vitae turpis tincidunt, varius dolor in, dictum lectus. Aenean ac ornare augue, ac facilisis purus. Sed leo lorem, molestie sit amet fermentum id, suscipit ut sem. Vestibulum orci arcu, vehicula sed tortor id, ornare dapibus lorem. Praesent aliquet iaculis lacus nec fermentum. Morbi eleifend blandit dolor, pharetra hendrerit neque ornare vel. Nulla ornare, nisl eget imperdiet ornare, libero enim interdum mi, ut lobortis quam velit bibendum nibh.
%
%Morbi tempor congue porta. Proin semper, leo vitae faucibus dictum, metus mauris lacinia lorem, ac congue leo felis eu turpis. Sed nec nunc pellentesque, gravida eros at, porttitor ipsum. Praesent consequat urna a lacus lobortis ultrices eget ac metus. In tempus hendrerit rhoncus. Mauris dignissim turpis id sollicitudin lacinia. Praesent libero tellus, fringilla nec ullamcorper at, ultrices id nulla. Phasellus placerat a tellus a malesuada.

%------------------------------------------------

%\section*{Conclusion}

%Fusce in nibh augue. Cum sociis natoque penatibus et magnis dis parturient montes, nascetur ridiculus mus. In dictum accumsan sapien, ut hendrerit nisi. Phasellus ut nulla mauris. Phasellus sagittis nec odio sed posuere. Vestibulum porttitor dolor quis suscipit bibendum. Mauris risus lectus, cursus vitae hendrerit posuere, congue ac est. Suspendisse commodo eu eros non cursus. Mauris ultrices venenatis dolor, sed aliquet odio tempor pellentesque. Duis ultricies, mauris id lobortis vulputate, tellus turpis eleifend elit, in gravida leo tortor ultricies est. Maecenas vitae ipsum at dui sodales condimentum a quis dui. Nam mi sapien, lobortis ac blandit eget, dignissim quis nunc.
%
%\begin{enumerate}
%\item First numbered list item
%\item Second numbered list item
%\end{enumerate}
%
%Donec luctus tincidunt mauris, non ultrices ligula aliquam id. Sed varius, magna a faucibus congue, arcu tellus pellentesque nisl, vel laoreet magna eros et magna. Vivamus lobortis elit eu dignissim ultrices. Fusce erat nulla, ornare at dolor quis, rhoncus venenatis velit. Donec sed elit mi. Sed semper tellus a convallis viverra. Maecenas mi lorem, placerat sit amet sem quis, adipiscing tincidunt turpis. Cras a urna et tellus dictum eleifend. Fusce dignissim lectus risus, in bibendum tortor lacinia interdum.

%----------------------------------------------------------------------------------------
%	BIBLIOGRAPHY
%----------------------------------------------------------------------------------------

\bibliographystyle{unsrt}

\bibliography{sample}

%----------------------------------------------------------------------------------------

\end{document}