\documentclass[11pt]{article} % use larger type; default would be 10pt

\usepackage[utf8]{inputenc} % set input encoding (not needed with XeLaTeX)

%%% Examples of Article customizations
% These packages are optional, depending whether you want the features they provide.
% See the LaTeX Companion or other references for full information.

%%% PAGE DIMENSIONS
\usepackage{geometry} % to change the page dimensions
\geometry{letterpaper} % or letterpaper (US) or a5paper or....
% \geometry{margin=2in} % for example, change the margins to 2 inches all round
% \geometry{landscape} % set up the page for landscape
%   read geometry.pdf for detailed page layout information

\usepackage{graphicx} % support the \includegraphics command and options

% \usepackage[parfill]{parskip} % Activate to begin paragraphs with an empty line rather than an indent

%%% PACKAGES
\usepackage{booktabs} % for much better looking tables
\usepackage{array} % for better arrays (eg matrices) in maths
\usepackage{paralist} % very flexible & customisable lists (eg. enumerate/itemize, etc.)
\usepackage{verbatim} % adds environment for commenting out blocks of text & for better verbatim
\usepackage{subfig} % make it possible to include more than one captioned figure/table in a single float
% These packages are all incorporated in the memoir class to one degree or another...

%%% HEADERS & FOOTERS
\usepackage{fancyhdr} % This should be set AFTER setting up the page geometry
\pagestyle{fancy} % options: empty , plain , fancy
\renewcommand{\headrulewidth}{0pt} % customise the layout...
\lhead{}\chead{}\rhead{}
\lfoot{}\cfoot{\thepage}\rfoot{}

%%% SECTION TITLE APPEARANCE
\usepackage{sectsty}
\allsectionsfont{\sffamily\mdseries\upshape} % (See the fntguide.pdf for font help)
% (This matches ConTeXt defaults)

%%% ToC (table of contents) APPEARANCE
\usepackage[nottoc,notlof,notlot]{tocbibind} % Put the bibliography in the ToC
\usepackage[titles,subfigure]{tocloft} % Alter the style of the Table of Contents
\renewcommand{\cftsecfont}{\rmfamily\mdseries\upshape}
\renewcommand{\cftsecpagefont}{\rmfamily\mdseries\upshape} % No bold!

%%% END Article customizations

%%% The "real" document content comes below...

\title{Asynchronous Chat: \\ Comparing REST-ful and Protocol Buffer Implementations}
\author{Serena Booth, Michelle Cone, Nicholas Mahlangu, Tianyu Liu}
\date{March 2, 2016} % Activate to display a given date or no date (if empty),
         % otherwise the current date is printed 

\begin{document}
\maketitle

\section{Hint from Waldo on how to write paper}
%\begin{itemize}
%\item What are the interfaces that are being looked at?
%\item Assumed environment?
%\item What are you trying to conserve?
%\item What you willing to waste?
%\item What are the failure conditions and how are you going to deal with them?
%\end{itemize}

\section{Introduction}

REST-ful communication protocols have taken the web by storm. In this design specification, we compare this exceedingly popular communication protocol with alternate protocol buffers,  having created a web application to facilitate online chat between users and groups using each of these communication protocols. 

\subsection{Interfaces Being Looked At}

\subsection{Assumed Environment}

\section{REST-ful Design}

\subsection{Long Polling} 

%TODO: JUSTIFICATION FOR WHY WE CHOSE THIS METHOD 

 In implementing our web chat application using the REST-ful communication protocol, we opted for a long polling approach, wherein the client opens a webpage which initiates frequent, constant communication with the server by means of a series of GET requests. 
 
We extend the long-polling approach to include a success handshake so as to confirm message receipt. In this way, if a message send operation is unsuccessful, we retry sending the message.

%TODO: INSERT DIAGRAM EXPLAINING THIS INTERACTION

 This long polling approach operates as follows: 

\begin{itemize}
\item The client opens a webpage which initiates a GET request using AJAX. This GET request is open until one of the following occur: (a) a success response is received or (b) an error response is received. 
\item In either (a) or (b), a completion script runs in which the AJAX request is dispatched again after a set amount of time. In the case of (a), there is no delay; in the case of (b), there is a 1000 millisecond delay. 
\item The server receives the GET request. It looks up messages for the particular client, based on a cookie passed in the request header, and sends an unread message back to the client, along with a success response. Further, the server sends the ids of the message returned in the response header. Lastly, the server updates the MySQL database of messages to indicate that this message is currently being sent. 
\item On (a), the success response, the client receives a message to display and an id corresponding to that message from the MySQL database. The client then dispatches an additional GET request containing the id of the message. 
\item On receiving the second confirmation GET request, the server updates the MySQL database to indicate that the message corresponding to a particular id has been received. 
\item If the server does not receive a success response a minute after sending a message, on the next client-initiated GET message corresponding to a request for new messages, the server updates the MySQL database to indicate that the formerly sent message was not received. 
\end{itemize} 

\subsection{Resources ``Conserved''} 

\subsection{Resources ``Wasted''}

\subsection{Failure Conditions}

The use of a REST-ful communication protocol lends this distributed system to failstop, crash, receive-omission, send-omission, and general omission failures. 

\section{Protocol Buffer Design}

\subsection{Resources ``Conserved''} 

\subsection{Resources ``Wasted''}

\subsection{Failure Conditions}

\section{Approach Comparison} 

\section{Conclusion} 

\end{document}
